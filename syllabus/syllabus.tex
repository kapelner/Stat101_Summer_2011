\documentclass[12pt]{article}

\usepackage[margin=0.95in]{geometry}
\usepackage{hyperref}
%\usepackage{amssymb}
%\usepackage{amsmath}

\title{Statistics 101 Summer 2011 \\ Course Syllabus \\ \vspace{0.5cm} \small{The Wharton School of the University of Pennsylvania}}
\author{Adam Kapelner}

\begin{document}
\maketitle

\begin{itemize}
\item[Instructor:]  Adam Kapelner
\item[Office:] 431.4 JMHH 
\item[Contact:] \url{kapelner@wharton.upenn.edu} (215-573-0531)
\item[Classes meet:] F55 JMHH Mon, Tues, Wed, Thurs 9--10:35AM May 23 - June 30
\item[Office Hours:] TBA
\end{itemize}

\section*{Course Overview}

This course develops ideas for helping to make decisions based on data. Some of the
following material will be covered: 

\begin{itemize}
\item Basic Set Theory
\item Counting Methods
\item Basic Probability Theory and Modeling
\item Random Variables, expectation, variance, covariance
\item Summary Statistics
\item Law of Large Numbers, The Central Limit Theory, and the Normal Distribution
\item Confidence Intervals, Hypothesis Testing, $p$-values
\item Introduction to Linear Regression
\item Data Displays (boxplots, histograms, etc)
\end{itemize}

The course does not dwell on the details of computation �-- its main focus is on understanding a few deep concepts and interpreting data and statistical results. 

This course is in a business school. Thus, practical examples will be drawn from business, finance, marketing, etc. Rudimentary knowledge of these fields will help, but is not required.

\section*{Course Materials}

\paragraph{Textbook:} Statistics for Business, Stine and Foster, 2011 (there will be additional readings as well, but this will be the main, required text)
\paragraph{Computer Software:} \textbf{JMP 8}, available at \url{http://upenn.onthehub.com} (3-year
license for \$59.95) Make sure to select your platform (Windows/Mac).
The software will also be used in Stat 102. If you have a Mac or any version of
Windows you should have no problems. We do \textit{not} recommend not buying a copy and relying on public Wharton computers. This will be inconvenient. We will also be using \texttt{R} which is a free, open source statistical programming language and console. You can download it from: \url{http://cran.opensourceresources.org/}. I do not expect you to do \textit{any} programming. I will be giving you \texttt{R} code to run and expect you to interpret the results based on concepts explained during the course.
\paragraph{Calculator:} It is not required but \textit{highly recommended} to have a TI-83, 83+, 84, 84+, or 89. I highly recommend buying a TI-89; I have been using it ever since high school (that's almost 10 years) and the dividends it pays just keep getting bigger. It can be used to check your answers on a lot of the statistical tests we will be doing in this course and it can solve calculus as a bonus. I see this on \url{newegg.com} for \$140 with free shipping.

\section*{Assignments, Examinations, and Grading Policy}

\subsection*{Homework}

There will be five homework assignments. Homeworks will be assigned on webCafe and will usually be due a week later in class. Homework will be graded out of 100. Homework must be neat and stapled. There will be a 15 point bonus for typing up your homework using the \LaTeX typesetting system (I will talk about this in class).

Late homework will be penalized 10 points per day for a maximum of three days. Do not ask for extensions; just hand in the homework late. Late homework can be put in my mail slot in the Department of Statistics, 4th floor of JMHH. After three days, credit will not be given because I will post the solutions.

Graded homework will be returned to the box labeled ``Stat 101 Summer'' which will be located in front of the conference rooms in the Department of Statistics. Scores for homeworks are finalized one week after the graded copies are handed back. Thereafter there will be no changes and no re-grading. Do not delay checking your graded homeworks to the end of the semester.

Homework is the \textit{most} important part of this course. Success in Statistics and Mathematics courses comes from experience in working with and thinking about the concepts. It's kind of like weightlifting; you have to lift weights to build muscles.

You are encouraged to seek help from the instructor if you have questions. You may also work with and help each other. You must, however, submit your own solutions, with your own write-up and in your own words. There can be no collaboration on the actual \textit{writing}. Failure to comply will result in severe penalties. The university honor code is something we take seriously and we send people to the Dean every semester for violations.

\pagebreak 

\subsection*{Examinations}

\begin{itemize}
\item One midterm examination will be held Mon, June 13, 6:00-8:00pm
\item The final examination will be Thurs, June 30, 6:00-8:00pm
\end{itemize}

The room assignments will be announced on webCafe closer to the exam dates. During examinations strict rules will be in effect with regard to honor code.

\subsection*{Grading and Grading Policy}

Your course grade will be calculated as follows: 20\% homework, 30\% midterm examination, and
50\% final examination. (What this means will be discussed during the semester.)


\section*{Use of webCafe}

Statistics 101 is using webCafe. You can gain access by going to
\url{webcafe.wharton.upenn.edu} and following the link to ``STAT'' and then to your
section. All materials for this course will be distributed and managed via this website,
and you will be able to monitor your grade entries throughout the semester.
An important feature of webCafe is the discussion board where everybody can place
questions and comments. We will be using it extensively for answering your questions
about homeworks, exams and scheduling. You are urged to go here first to see whether
your question has already been asked and answered, and, if not, to place your question so
it can be answered once for everybody.

Note for non-Wharton students: If you do not have a Wharton computing account, you
will need to establish one to access the website. The account also provides access to the
computing labs in Wharton and to the intranet. To get an account, on or after the first day
of classes, go to \url{accounts.wharton.upenn.edu}. After you have obtained your
account, allow up to 12 hours for activation.


\end{document}
